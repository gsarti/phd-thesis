\chapter*{Acknowledgements}

\vspace{-0.3cm}

% Intro
\noindent 

Outsiders to the academic world might see the PhD as a lonely, long-winding path in the pursuit of knowledge. On the contrary, looking back over the past four years, it's clear to me that it truly takes a village to raise happy, fulfilled PhD students. The following few pages pay homage to my own little village, to which I owe the wonderful experience that is now reaching its conclusion.

This thesis would not have been possible without the thoughtful guidance of my supervisors, Arianna Bisazza, Malvina Nissim and Grzegorz Chrupa\l{}a. Arianna, I cannot thank you enough for giving me a chance to embark on this journey in Groningen. Your outstanding curiosity and passion for research were a constant source of motivation for me to keep asking difficult questions, dig deeper into promising directions, and never stop learning new things. I cherish the time we spent together, at the office debating promising new directions and asking ourselves ``What does it take to make interpretability useful in practice?'', playing Codenames at lab retreats, or enjoying amazing meals at conferences all around the world. Your keen eye for designing and executing compelling research agendas and your genuine attention to the well-being of everyone in the lab have been great sources of inspiration for the kind of scientist I strive to become. Malvina, I cannot imagine my PhD journey without your unwavering support. Your cheerfulness and excitement have always made challenging moments feel lighter, and I thank you for being the voice of reason that often helped me strike a balance between work and life matters. Working with you taught me that research is, in large part, about \textit{community}, and that the time invested in sustaining and growing the communities we genuinely care about is time well spent. Your commitment to making advances in our field accessible to the general public sparked my interest in science communication, and your work with various communities to ensure that your research can be inclusive and useful motivated me to adopt a user-centric perspective in my interpretability work. I hope our paths will cross again, especially if it's for a chat over a cecina in Pisa, or at the next CLiC-it! Grzegorz, I am grateful that you agreed to join us as my co-supervisor for this journey. Thank you for consistently contributing interesting perspectives during our InDeep meetings and for your invaluable support for our \textsc{PECoRe} and QE4PE projects. I am still looking forward to tasting one of your famed espressos!

I am also thankful to the members of my reading committee, Profs. Barbara Plank, Ivan Titov and Michael Biehl, for agreeing to spend some time on this (admittedly lengthy) manuscript and providing me with valuable feedback to improve it. Also, thank you to Profs. Eva Vanmassenhove, Rik van Noord and Jelle Zuidema for joining my defense committee and bringing in your expertise on machine translation, bias and interpretability.

My research project was funded and carried out as part of the NWO InDeep consortium, led by Prof. Jelle Zuidema. A big thank you to all professors --- Afra Alishahi, Anstke Fokkens, Ashley Burgoyne, Iris Hendrickx, Louis ten Bosch, Martijn Bentum, Tom Lenz --- and to my fellow PhDeeps --- Charlotte Pouw, Gaofei Shen, Hosein Mohebbi, Marcel Vélez, Jane Arleth dela Cruz, Jonathan Kamp --- for always making me feel part of a thriving research community working together to make deep learning models more transparent for applications in speech, text and music. Thank you also to the honorary UvA PhDeeps --- Michael Hanna, Marianne de Heer Kloots, Oskar van der Wal --- who were always involved in our meetings, and to Rudmer Canjels for handling the consortium activities and funding our (occasional) virgin mojito.

The GroNLP group at the University of Groningen has been my research home for the past four years, and remains one of the most friendly and welcoming work environments I have seen to this day. A big thank you to all my wonderful colleagues, past and present --- Gertjan van Noord, Johan Bos, Martijn Wieling, Gosse Bouma, Andreas van Cranenburgh, Antonio Toral, Federico Pianzola, Khalid Al Khatib, Defne Abur, Frank Tsiwah, Lisa Bylinina, Kun He, Huiyuan Lai, Wietse de Vries, Martijn Bartelds, Chunliu Wang, Teja Rebernik, Prajit Dhar, Masha Medvedeva, Raoul Buurke, Silvia Stopponi, Thomas Tienkamp, Iris Schepers, Hedwig Sekeres, Xiao Zhang, Greta Zella, Reihaneh Amouie, Noa Visser, Katharina Polsterer, Xiaoyan Yang, Yun Hao, Ze Yu, Yanan Wu, Yongjian Chen, Nastja Shaboltas, Kyo Gerrits, Xiaolu Wang, Susan Lotz, Shaozhen Shi, Sijbren van Vaals, Sara Nabhani, Dan Mu, Ruhi Mahadeshwar, Valentine Lucquiault, Ezgi Ba\c{s}ar and Gijs Danoe. A special thanks to fellow members of the InCLow team --- Jaap Jumelet, Jirui Qi, Francesca Padovani, Yuqing Zhang and Yevgen Matusevych --- with whom I had a chance to share our cozy office, discuss interesting work at our reading groups, or spend some quality time together in Suzhou. Francesca and many other colleagues --- Pritha Majumdar, Leonidas Zotos, Franziska Pannach and Marìa Pilar Uribe Silva --- joined the group after me, bringing in a fresh breath of air and enthusiasm for spending time together in Groningen. Thank you all for having filled my days in Groningen with nice chats and get-togethers! Ahmet \"{U}st\"{u}n, I'm happy we stayed in touch after the end of your PhD and for the interesting discussions, ranging from hardcore technical topics to Muteh\c{s}em Y\"{u}z{\i}l over a coffee or a tantuni! 

In the many publications that filled the past few years, I even had the chance to collaborate with some colleagues towards interesting research ideas --- Lukas Edman, Ana Guerberof, Tommaso Caselli and Daniel Scalena. Joachim Lukas, while we were sharing a room in Abu Dhabi you revealed your bropensity for joking around, and since then our (mostly non-research) conversations have always been a wild ride. Looking forward to many years of hanging out at conferences around the world! Ana, thank you for always being a cheerful presence in the office and for always bringing your no-nonsense attitude to keep our project meetings grounded in reality. Your expertise on post-editing and translation workflows has been essential to ensuring the soundness of user studies in this thesis. Tommaso, I enjoyed our collaboration on the rebus paper, but I enjoyed the spaghettate partigiane at your place on April 25th even more! Looking forward to collaborating on making attribution useful for socially relevant applications. Last but not least, Daniel, I am grateful to have had the chance to collaborate with you on so many interesting projects and to share lovely memories in Singapore, Miami and Boston. Seeing you grow into a talented researcher has been very inspiring, and I'll always be on the lookout for your next outstanding interpretability paper!

Finally, the GroNLP acknowledgements would not be complete without thanking the many people that left their mark in our group with their visits --- Arianna Graciotti, Arianna Muti, Daniela Occhipinti, Beatrice Savoldi, Serena Coschignano, Giulia Rambelli, Elena Sofia Ruzzetti, Jani\c{c}a Hackenbuchner, Anaís Almendra, Akari Haga, Saad Amin, Okky Ibrohim --- and all the students whom I had the chance to meet and supervise as part of the Advanced NLP course and the LCT masters' program, in particular Ludwig Sickert, Qiankun Zheng, Konstantin Chernyshev and Khondoker Islam. I also acknowledge the support of the Center of Information Technology at the University of Groningen, which supported my research by providing free access to the Peregrine and Habrók clusters, and the SURF cooperative, which granted me free access to their Snellius supercomputer.

Aside from our group in Groningen, I am fortunate to have established connections with many outstanding researchers worldwide. First, a warm thank you to the colleagues at the ItaliaNLP Lab --- Felice dell'Orletta, Alessio Miaschi, Chiara Alzetta, Giulia Venturi, Dominique Brunato and Cristiano Ciaccio --- who introduced me to NLP research in the year 2019 BCE (Before the ChatGPT Era), and who remain to this day close collaborators for Italian NLP projects. I am also thankful to the Amazon Translate team, and in particular to Georgiana Dinu, Maria Nadejde and Xing Niu for welcoming me into their team during my first year of PhD and for helping me discover the exciting world of industrial research. In the past few years, I have been privileged to collaborate with outstanding fellow PhDs --- Nils Feldhus, Hosein Mohebbi, Javier Ferrando and Vilém Zouhar. Nils, Hosein, Javier and Vilém, our discussions and your thoughtful support laid the foundation for some of my most impactful works to date. You are all incredibly talented scientists and I hope to have the chance to work together again someday! Alongside Hosein, I also want to thank all the other co-organizers of BlackboxNLP 2025 --- Yonatan Belinkov, Aaron Mueller, Najoung Kim, Hanjie Chen, Dana Arad and Martin Tutek --- for giving me a chance to be involved in the organization of this year's edition of the workshop, and for working together efficiently to make it a success. A special shoutout to Dana, whose in-person problem-solving skills were fundamental to ensuring everything ran smoothly on the day of the event. I want to also sincerely thank Luca Bortolussi --- who always kept me in the loop and included me in the initiatives and activities of the Trieste AI community --- and the many members of the UniTS AILab --- in particular Emanuele Ballarin, Sara Candussio, Davide Scassola, Gaia Saveri, Francesca Cairoli, Francesco Giacomarra --- for always making me feel welcome in my moments back in Trieste. Finally, I want to thank David Bau and the members of the BauLab for their warm welcome this summer at NEMI. David, you are an inspiring researcher (coincidentally, the cover of this thesis was inspired by your 3D-printed aperiodic monotiles!), and I thank you for sharing my vision for an open interpretability ecosystem that benefits everyone. I am excited to join you soon in Boston and to keep advancing the frontier of interpretability research together.

Beyond work, I want to thank everyone else who made these years unforgettable. First, friends in Germany --- Sergei Kolesnikov, Anastasiia Nikitina, Gleb Tomachevski, Perancha Domingo, Victor Martinez, Priscilla Oh --- who made me feel right at home during my commuter life between Groningen and Bremen. Sergei, thank you for being an excellent teacher and getting me from zero to hero in Russian in no time! Our dinners and board game evenings with Lera and Nastia, and our soccer matches with Gleb are among my happiest memories of these years. Thanks also to all my Italian friends all around the world --- Mario Giulianelli, Giuseppe Attanasio, Andrea Santilli, Francesco Cicala, Salvatore Milite, Andrea Gasparin, Leonardo Stincone, Andrea Lorenzon --- I am glad we are still in touch, to discuss exciting new projects, or simply to update each other about our lives. Thank you also to my long-time friends from Galilei --- Ethan Turco, Elia Miraz, Jimmi d'Ambrosi, Enrico Pieri, Lorenzo Macor, Simone Ramazzotti --- with whom I still share interesting conversations over a Negroni pitcher after more than ten years apart.

This section would not be complete without thanking some very special people. First, my amazing paranymphs, Otje Minnema and Sara Gemelli, who have been two of my closest friends throughout my PhD path, and an essential help in preparing for my defense. Your presence kept me grounded when the stress and the traveling were taking a toll, and you made Groningen feel like home from the very beginning of this journey --- I loved our time together in the Noordernplantsoen, our trips to Schier, and our dinners on the Pannekoekschip. Lunga vita al triumpersonato! Mattia, despite our physical distance and our busy schedules, I never felt distant from you for a second in all these years, knowing that at the next meeting we would both feel right at home with each other, as if we never left Trieste. Thank you for being a steady presence in my life, and for teaching me every day what friendship is really about. 

I want to thank my family for always being there to cheer for my successes and comfort me in challenging times from Trieste, Terni and Rome. Dad, mom, Barbara and Diego --- your love is the solid foundation at the base of all my successes, and your unwavering support has given me, time and time again, the confidence to leave my safe harbor and set sail towards unknown waters. Words fail me in expressing how much this matters to me.

Dulcis in fundo, I want to express my heartfelt appreciation to my partner, Valeriya, who has been my constant through every variable and my solution for every problem throughout this journey. Your ever-loving presence and help made even the most challenging moments feel easy. Thank you for being my steady ground, my voice of reason and the love of my life.